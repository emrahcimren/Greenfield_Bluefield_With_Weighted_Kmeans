%%
%% Author: ecimren
%% 4/6/2019
%%

% Preamble
\documentclass[12pt]{article}
% Use latex macro
\input latexmacro
\standard

\oddsidemargin=0in%
\evensidemargin=0in %
\topmargin =-.5in %
\textheight=8.5in %
\textwidth=6.4in %

% Packages
\usepackage{amsmath}
\usepackage{natbib}
\usepackage{graphicx}
\pagenumbering{arabic}


% Document
\begin{document}
    \double

    %\begin{center}

     %   \Large{\bf Binary Program for Customer to Distribution Center Allocation}

       % \bigskip
        %\normalsize{Emrah Cimren \\
        %cimren.1@gmail.com}

        %\bigskip
        %\normalsize\today
        %\bigskip

    %\end{center}

    %\begin{abstract}
     %   This document describes customer to distribution center allocation formulation.
    %\end{abstract}

    %\section{Model}

    We first define sets and parameters and then define model variables.

    Let $C$ be set of customers and $D$ be set of distribution centers.

    Let $d_{j}$ be total demand for the customer $j \in C$.

    Let $a_{ij}$ be the distance from the center $i \in D$ to customer $j \in C$.

    Let $\alpha_{i}$ be the maximum distance covered by the center $i \in D$.

    Let $u^{-}_{i}$ be the minimum number of stores can be allocated to center $i \in D$.

    Let $u^{+}_{i}$ be the maximum number of stores can be allocated to center $i \in D$.

    Let $y_{ij}$ be the binary variable for assigning center $i \in D$ to customer $j\in C$ where

    \begin{equationarray}{rlllll}
    y_{ij}&=&\left\{
    \begin{array}{ll}
    1, & \mbox{if center }i\in D \mbox{ is assigned to customer }j \in C,\nonumber\\[5pt]
    0, & \mbox{otherwise.} \nonumber
    \end{array}
    \right.\nonumber
    \end{equationarray}

    The following is the binary program for the distribution center and customer allocation.

    \bigskip

    \begin{equationarray}{rrrlll}
        \mbox{minimize} \sum_{i\in D}\sum_{j\in C} d_{j}a_{ij}y_{ij} \label{model:objective}
    \end{equationarray}

    \vspace{-6pt}

    \begin{equationarray}{rrrlll}

        \mbox{subject to} & \sum_{i \in D}y_{ij} & = & 1, & \forall j\in C, \label{model:each customer to center}\\[18pt]
                        & a_{ij} y_{ij} & \le & \alpha_{i}, & \forall i \in D, j\in C, \label{model: max distance}\\[18pt]
                        & \sum_{j \in C} y_{ij} & \ge & u^{-}_{i}, & \forall i \in D, \label{model: min customers by center}\\[18pt]
                        & \sum_{j \in C} y_{ij} & \le & u^{+}_{i}, & \forall i \in D, \label{model: max customers by center}\\[18pt]
                        & y_{ij} & \in & \{0, 1\}, & \forall i \in D, j\in C. \label{model: variable binary}
    \end{equationarray}

    The objective of the model (\ref{model:objective}) minimizes the total weighted distance.
    Constraint (\ref{model:each customer to center}) ensures that each customer is allocated to a center.
    There exists a maximum distance can be covered by each center (\ref{model: max distance}).
    Each center has minimum and maximum number of allocated customers as in (\ref{model: min customers by center})
    and (\ref{model: max customers by center}), respectively.
    Center to customer allocation is binaries as in (\ref{model: variable binary}).

\end{document}